\begin{BlueBox}
    \vskip-1cm
    \begin{block}{\BHead{Approach}}
        \begin{itemize} 
            \item Tagged Ozone Production Potentials (TOPPs) \citep{Butler:2011} calculated over 7 days for VOCs common to urban environments. \vspace{5mm}
            \item Following mechanisms are compared. MCM v3.2 is the reference mechanism.
                {
                    \setstretch{1.1} \normalsize
                    \begin{table}[htp]
                        \begin{center}
                            \begin{tabular}{P{7cm}P{7cm}P{7cm}}
                                MCM v3.2 & MCM v3.1 & CRI v2 \\
                                RADM2 & RACM & RACM2 \\
                                MOZART-4 & CBM-IV & CB05
                            \end{tabular}
                        \end{center}
                    \end{table}
                } \vspace{5mm}
            \item \ce{O_x} (= \ce{O3 + NO2}) production is allocated back to the emitted VOC by tagging all organic degradation products produced from the VOC. This is done by labelling the organic products with the emitted VOC's name. \vspace{5mm}
            \item Daily \ce{O_x} production normalised by the VOC emission giving the ratio of \ce{O_x} production per molecule of VOC (TOPP).
        \end{itemize}
    \end{block}
\end{BlueBox}
