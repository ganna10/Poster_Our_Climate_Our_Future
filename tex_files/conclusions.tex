\begin{BlueBox}
    \vskip-1cm
    \begin{block}{\BHead{Conclusions}}
        \begin{itemize}
            \item More explicit mechanisms show larger \ce{O_x} production than those with a more streamlined approach. As VOCs break down 
            quicker in less-explicit mechanisms, there is less \ce{O_3} produced. \vspace{5mm}

            \item Aromatic chemistry is represented very differently between the mechanisms, resulting in a wide spread of TOPP values between 
            the mechanisms. This is not so surprising as aromatic chemistry is not fully understood and subject to very large uncertainties in the
            overall chemistry. The approach taken by the RACM mechanism, in particular, differs substantially from all the other mechanisms 
            resulting in the unrealistic net consumption of \ce{O_x} on the first two days.\vspace{5mm}

            \item The first day TOPPs are similar for a number of VOCs however there are more differences when looking at their temporal 
            profile. \vspace{5mm}
        \end{itemize}        
    \end{block}
\end{BlueBox}
